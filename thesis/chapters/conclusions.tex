In this thesis, we have explored a wide range of algorithmic approaches to solve the Travelling Salesman Problem, including heuristic, metaheuristic, exact, and matheuristic methods. 
The objective was not only to assess the quality of the solutions generated by each method but also to evaluate their computational efficiency across different instance sizes and time budgets.

Among the heuristic approaches, the Nearest Neighbor algorithm combined with the 2-opt local search proved to be an effective baseline, offering rapid improvements with minimal computational cost. However, its limitations became evident when compared to more sophisticated techniques.
In the field of metaheuristics, Tabu Search emerged as the most effective strategy. Thanks to its adaptive memory-based mechanism and dynamic tenure control, it consistently outperformed Variable Neighborhood Search, particularly in scenarios with extended time budgets. 
The performance profiles confirmed its ability to produce high-quality solutions across a broad range of instances, justifying the added complexity of its design.

Regarding exact methods, the Branch-and-Cut algorithm demonstrated superior performance and robustness compared to Benders Decomposition. 
While warm starts based on Tabu Search did not yield significant improvements-likely due to their overhead-lighter strategies such as 2-opt proved to be beneficial, especially in combination with Benders.

The matheuristic strategies offered an interesting compromise between heuristic guidance and mathematical rigor. Among them, Hard Fixing achieved the best results by effectively narrowing the search space while allowing diversification through probabilistic fixing schemes. Local Branching, while conceptually appealing, showed more limited effectiveness in comparison.
Overall, the results highlight the advantages of hybrid and adaptive approaches in solving NP-hard problems like the TSP. 

Future work could investigate the integration of machine learning techniques for automatic parameter tuning, such as Optuna \cite{optuna}, or explore new combinations of methods to further enhance performance on large-scale instances.
