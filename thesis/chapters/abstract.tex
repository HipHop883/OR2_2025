The Travelling Salesman Problem (TSP) is a fundamental and widely studied optimization problem in the field of Operations Research. It was originally formulated to find the shortest possible route that visits a set of cities exactly once and returns to the starting point. Despite its simple formulation, the TSP is NP-hard, making it computationally challenging to solve for large instances.

Over the decades, a variety of algorithmic approaches have been developed to tackle the problem, ranging from exact methods to heuristics and metaheuristics. Among these, Concorde represents the state-of-the-art in exact TSP solvers, leveraging advanced optimization techniques. However, such high-performance tools are often complex and computationally demanding.

This study aims to explore and compare a selection of more accessible algorithmic strategies, including classical heuristics, metaheuristics and exact methods. Through implementation and experimental evaluation, the work provides insights into their practical performance and highlights the trade-offs between computational efficiency and solution quality in solving the TSP.
