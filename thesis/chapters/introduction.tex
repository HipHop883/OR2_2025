The Travelling Salesman Problem (TSP) is one of the most well-known and extensively studied problems in the field of Operations Research. 
First mathematically formulated in the 19th century by the mathematicians William Rowan Hamilton and Thomas Kirkman \cite{Biggs1986Graph}, the TSP began to attract significant scientific attention in the 1950s. 
It asks for the shortest possible route that visits each node in a graph exactly once and returns to the starting point—essentially, the shortest Hamiltonian cycle. 
In 1972, Richard M. Karp proved that the TSP is NP-hard, implying that the computation time needed to solve it can grow exponentially with the size of the input \cite{Karp1972}. Despite this complexity, 
many exact and approximate algorithms have been developed over the years, offering different trade-offs between optimality and efficiency. The current state-of-the-art in solving large-scale 
instances of the problem is the Concorde solver, developed by Applegate, Bixby, Chvátal, and Cook, which uses advanced techniques in combinatorial optimization, linear programming, and cutting-plane methods \cite{Applegate2006}. 
The goal of this thesis is not to propose new methods capable of competing with Concorde, but rather to explore, implement, and compare basic yet effective algorithmic approaches that are computationally accessible and still capable of producing meaningful results.