The Travelling Salesman Problem (TSP) has long served as a benchmark in combinatorial optimization and computational complexity. Although its origins can be traced back to the 19\textsuperscript{th} century with the work of Hamilton and Kirkman~\cite{Biggs1986Graph}, it was not until the mid-20\textsuperscript{th} century that the problem gained serious scientific traction. The TSP asks whether a minimal-cost cycle exists that visits each node in a graph exactly once and returns to the starting point—a problem later proven to be NP-hard by Richard M. Karp in 1972~\cite{Karp1972}.

The relevance of the TSP extends well beyond its theoretical appeal: it arises naturally in domains such as logistics, manufacturing, genome sequencing, and circuit design. Solving it efficiently has therefore been a longstanding goal in both academic and industrial research. Over the years, a wide variety of algorithms have been proposed, ranging from exact methods based on integer programming to heuristic and metaheuristic approaches.

One of the most advanced exact solvers is Concorde, which incorporates sophisticated techniques such as cutting planes, branch-and-bound, and polyhedral combinatorics~\cite{Applegate2006}. While extremely powerful, tools like Concorde are complex to implement and computationally intensive.

This thesis focuses on more accessible algorithmic strategies for solving the TSP. Rather than aiming for optimality at any cost, the goal is to explore and compare classic heuristics and metaheuristics in terms of their practicality, computational efficiency, and solution quality across a range of problem instances.


\section{Problem Formulation}

Let us consider an undirected graph $G = (V, E)$, where $V$ is the set of $|V| = N$ nodes (or vertices), 
and $E$ is the set of $|E| = M$ edges. A \textit{Hamiltonian cycle} of $G$, denoted by $G^* = (V, E^*)$, 
is a subgraph whose edges form a cycle that visits each node $v \in V$ exactly once.
We define a cost function $c : E \rightarrow \mathbb{R}^+$ that assigns a non-negative cost $c_e = c(e)$ to each edge $e \in E$. 
The objective of the Travelling Salesman Problem (TSP) is to find a Hamiltonian cycle in $G$ that minimizes the total cost, defined as:
\[
\text{cost}(\text{cycle}) := \sum_{e \in \text{cycle}} c(e)
\]

This problem can be formulated as an Integer Linear Programming (ILP) model. We define binary decision variables $x_e$ to indicate whether edge $e$ is included in the optimal cycle:
\[
x_e =
\begin{cases}
1 & \text{if } e \in E^* \\
0 & \text{otherwise}
\end{cases}
\quad \forall e \in E
\]

The ILP model is as follows:
\begin{align}
\text{minimize} \quad & \sum_{e \in E} c_e x_e \label{eq:ilp_obj} \\
\text{subject to} \quad & \sum_{e \in \delta(v)} x_e = 2 \quad \forall v \in V \label{eq:ilp_degree} \\
& \sum_{e \in E(S)} x_e \leq |S| - 1 \quad \forall S \subset V,\, v_1 \in S \label{eq:ilp_sec} \\
& x_e \in \{0, 1\} \quad \forall e \in E \label{eq:ilp_binary}
\end{align}

Constraint~\eqref{eq:ilp_degree} ensures that each node has degree 2, i.e., it is entered and exited exactly once. 
However, these constraints alone are not sufficient to guarantee a single, valid Hamiltonian cycle: the solution may consist of multiple disjoint cycles. 
To prevent this, the model includes constraint~\eqref{eq:ilp_sec}, known as the \textit{Subtour Elimination Constraints} (SECs), 
which ensure that the solution forms a single connected cycle where every node $v \neq v_1$ is reachable from $v_1$.

Despite their theoretical importance, SECs are defined for every subset $S \subset V$ containing $v_1$, and their number is exponential in $N$. 
Including all of them simultaneously is therefore computationally infeasible.

Throughout this report, all pseudocode and algorithmic approaches will assume an undirected, complete graph $G = (V, E)$ and a cost function $c : E \rightarrow \mathbb{R}^+$.
