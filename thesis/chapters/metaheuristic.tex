\section{Metaheuristics}

A metaheuristic is a high-level, abstract problem-solving strategy designed to efficiently explore and exploit large solution spaces 
in complex optimization problems. Unlike algorithms tailored to specific instances, metaheuristics are general-purpose frameworks 
that guide subordinate heuristics—leveraging iterative improvement, intelligent search patterns, and adaptive behavior to find near-optimal solutions.

In our experiments, we employ metaheuristics that use the \textit{2-opt} algorithm as a foundational local search heuristic. 
Specifically, we take greedy solutions refined by 2-opt as initial configurations and further enhance them using one of two metaheuristic methods: 
\textbf{Tabu Search} and \textbf{Variable Neighborhood Search (VNS)}.

These strategies offer the ability to explore a broader search space than simple heuristics by incorporating mechanisms that allow 
occasional acceptance of worse solutions— thereby escaping local optima. We demonstrate that even relatively simple metaheuristic ideas, 
when combined with previously introduced heuristics, can yield solutions remarkably close to the optimal ones.

\section{Tabu Search}

Tabu Search is a metaheuristic developed to efficiently explore the solution space while avoiding local optima. 
Originally introduced by Fred Glover in 1986 and later formalized in 1989~\cite{Glover:TabuSearch}, it is well-suited for combinatorial problems like the Traveling Salesman Problem (TSP).

The method enhances a basic local search (in our case, 2-opt) by allowing non-improving moves and using a \textit{tabu list} to prevent cycling back to recently visited solutions.

\subsection{Search Strategy}

The search alternates between two phases:
\begin{itemize}
    \item \textbf{Intensification:} The algorithm performs 2-opt moves to locally improve the current solution.
    \item \textbf{Diversification:} When no improvement occurs for a specified number of iterations, a random move is introduced to escape local minima, 
        and the tabu tenure is increased to promote exploration.
\end{itemize}

\subsection{Implemented Mechanism}

In our implementation, the algorithm follows these steps:

\begin{enumerate}
    \item Generate an initial solution, either randomly or greedily.
    \item At each iteration, evaluate all possible 2-opt moves and select the best one not present in the tabu list.
    \item If the selected move does not improve the current solution, it is added to the tabu list.
    \item If a better solution is found and the tenure is greater than the minimum, the tenure is decreased to focus the search locally.
    \item If no improvement is made over a certain number of iterations, a random move is performed and the tenure is increased to enhance diversification.
\end{enumerate}

This results in a dynamic adjustment of the tabu tenure, based on the recent progress of the search, without relying on predefined functions or schedules.

\subsection{Tabu List Details}

Each tabu move corresponds to a 2-opt edge swap, and is represented by storing the two removed edges, $(p_i, p_{i+1})$ and $(p_j, p_{j+1})$. 
These moves are kept in the tabu list for a number of iterations defined by the current tenure. The list helps prevent the reversal of recent transformations 
and encourages the algorithm to explore new regions of the solution space.

\subsection{Pseudocode}

\begin{algorithm}
\caption{Tabu Search (simplified)}
\begin{algorithmic}
    \State $\text{sol} \gets$ initial solution
    \State $\text{best\_sol} \gets \text{sol}$
    \State $\text{tabulist} \gets \emptyset$
    \While{time limit not exceeded}
        \State $\text{move} \gets$ best 2-opt move not in tabu list
        \State Apply $\text{move}$ to $\text{sol}$
        \If{$\text{sol}$ is better than $\text{best\_sol}$}
            \State $\text{best\_sol} \gets \text{sol}$
            \If{tenure $>$ minimum}
                \State Decrease tenure
            \EndIf
        \Else
            \State Add $\text{move}$ to tabu list
            \State Increase counter for non-improving iterations
        \EndIf
        \If{counter exceeds threshold}
            \State Apply a random move
            \If{tenure $<$ maximum}
                \State Increase tenure
            \EndIf
        \EndIf
    \EndWhile
\end{algorithmic}
\end{algorithm}

The version of Tabu Search implemented here uses a simple yet adaptive strategy to balance exploration and exploitation. 
The dynamic adjustment of the tabu tenure—based on the quality of recent iterations—provides effective guidance through the solution space 
without introducing unnecessary complexity.